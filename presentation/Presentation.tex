%\documentclass{beamer}
%\usepackage{minijs}
%\usepackage{here}
%\mode<presentation>{
%  \usetheme{Antibes}
%  \usecolortheme{beaver}
%  \setbeamercovered{transparent}
%}
\documentclass[13pt,dvipdfmx]{beamer}
\usepackage[english]{babel}
% pdfの栞の字化けを防ぐ
\AtBeginDvi{\special{pdf:tounicode EUC-UCS2}}
% テーマ
\usetheme{AnnArbor}
\usepackage{caption}
\setbeamertemplate{navigation symbols}{} 
\usepackage{graphicx}
%\usepackage[dvipdfmx]{graphicx}
\usepackage{amsmath}
\usepackage{amssymb}
\usepackage{txfonts}
\usepackage{colortbl}
\renewcommand{\familydefault}{\sfdefault}
\renewcommand{\kanjifamilydefault}{\gtdefault}
\setbeamerfont{title}{size=\large,series=\bfseries}
\setbeamerfont{frametitle}{size=\large,series=\bfseries}
\setbeamertemplate{frametitle}[default][center]
\usefonttheme{professionalfonts} 

%1ページめ
\title{クラスタリング手法の評価}
\author{AF16009 池辺 颯一}
\institute{SIT}
\date{2018年12月12日}

\begin{document}
\begin{frame}\frametitle{}
 \titlepage
\end{frame}

\begin{frame}\frametitle{背景}
\begin{itemize}
 \item 情報化社会の発展によりデータ量が膨大になりつつある
 \item それらのデータを自動的に分類するクラスタリングに着目
\end{itemize}
\end{frame}

\begin{frame}\frametitle{目的}
\begin{block}{目的}
\begin{itemize}
 \item 
\end{itemize}
\end{block}
\vspace{4mm}
\begin{block}{目標}
\begin{itemize}
 \item 各クラスタリング手法について比較及び評価を行う
\end{itemize}
\end{block}
\end{frame}

\begin{frame}\frametitle{実験方法}
\begin{center}
\end{center}
\end{frame}

\begin{frame}\frametitle{実験評価方法}
\begin{center}
\end{center}
\end{frame}

\begin{frame}\frametitle{\Large クラスタリング手法}
 \begin{block}{eFCMA}
  $\underset{u,v,\pi}{\text{minimize}}$
  $\sum_{i=1}^C\sum_{k=1}^Nu_{i,k}||x_k-v_i||_2^2+\lambda^{-1}\sum_{i=1}^C\sum_{k=1}^Nu_{i,k}\log(\frac{u_{i,k}}{\pi_{i}})$\centering\\
 \end{block}
\begin{align*}
 d_{i,k}&=||x_{k}-v_{i}||_{2}^2, \\
 u_{i,k}&=\frac{\pi_{i}\exp(-\lambda||x_k-v_i||_2^2)}{\sum_{j=1}^C\pi_{j}\exp(-\lambda||x_k-v_j||_2^2)\quad,}\\
 v_{i}=\frac{\sum_{k=1}^Nu_{i,k}x_{k}}{\quad\sum_{k=1}^Nu_{i,k}},
 %%\alpha_{i}&=\frac{\sum_{k=1}^Nu_{i,k}}{\quad N}.
\end{align*}
\end{frame}

\begin{frame}\frametitle{\Large クラスタリング手法}
 \begin{block}{qFCMA}
  $\underset{u,v,\alpha}{\text{minimize}}$
  $\sum_{i=1}^C\sum_{k=1}^N(\alpha_{i})^{1-m}(u_{i,k})^m||x_k-v_i||_2^2+\frac{\lambda^{-1}}{m-1}\sum_{i=1}^C\sum_{k=1}^N(\alpha_{i})^{1-m}(u_{i,k})^m$\centering\\
 \end{block}
 \begin{align*}
  d_{i,k}&=||x_{k}-v_{i}||_{2}^2, \\
  u_{i,k}&=\frac{\alpha_{i}(1+\lambda(1-m)||x_i-v_k||_2^2)^\frac{1}{1-m}}{\quad\sum_{j=1}^C\alpha_{j}(1+\lambda(1-m)||x_j-v_k||_2^2)^\frac{1}{1-m}\quad,}\\
  v_{i}=\frac{\sum_{k=1}^N(u_{i,k})^mx_{k}}{\quad\sum_{k=1}^N(u_{i,k})^{m}},
 \end{align*}
\end{frame}

\begin{frame}\frametitle{\large クラスタリング手法}
 \begin{block}{sfcma}
  $\underset{u,v,\alpha}{\text{minimize}}$
  $\sum_{i=1}^c\sum_{k=1}^n(\alpha_{i})^{1-m}(u_{i,k})^m||x_k-v_i||_2^2$\\
  subject \; to \; $\sum_{i=1}^cu_{i,k}=1$\;,\;$\sum_{i=1}^c\alpha_{i}=1$\;and\;$u_{i,k}\in[0,1]$\quad$m>1$
 \end{block}
 \begin{align*}
  d_{i,k}&=||x_{k}-v_{i}||_{2}^2=\left(\sqrt{\sum_{\ell=1}^m (x_{k,\ell}-v_{i,\ell})^2}\;\right)^2,\\
  u_{i,k}&=\frac{1}{\sum_{j=1}^c\frac{\alpha_{j}}{\alpha_{i}}\left(\frac{d_{j,k}}{d_{i,k}}\right)^\frac{1}{1-m}},\quad
  v_{i}=\frac{\sum_{k=1}^n(u_{i,k})^mx_{k}}{\quad\sum_{k=1}^n(u_{i,k})^{m}},\\
 \end{align*}
\end{frame}


\begin{frame}\frametitle{使用する実データ}
\begin{center}
{\Large Yeast Data Set}
\begin{figure}
\end{figure}
\end{center}
\end{frame}

\begin{frame}\frametitle{実験結果}
\begin{center}
{\Large}
\begin{figure}
\end{figure}
\end{center}
\end{frame}

\begin{frame}\frametitle{考察}
\begin{itemize}
 \item 
\end{itemize}
\end{frame}

\begin{frame}\frametitle{まとめ}
\begin{itemize}
 \item 
\end{itemize}
\end{frame}

\end{document}