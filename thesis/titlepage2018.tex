\documentclass[a4j]{jarticle}
\thispagestyle{empty}
\usepackage{graphics}
\usepackage{color}
\usepackage{times}
\usepackage{mathptmx}
\usepackage{pifont}
\usepackage{tascmac}
\usepackage{oldgerm}
\usepackage{jtygm}
\usepackage{type1cm}
\advance\textheight by 1in
%\advance\textheight by \headheight
\advance\textheight by \headsep
\advance\textheight by \topskip
\advance\textheight by \footskip
\topmargin=-1in
%\headheight=0pt
\headsep=0pt
\topskip=0pt
\footskip=0pt
\advance\textwidth by 1in
\advance\textwidth by \marginparwidth
\advance\textwidth by \marginparsep
\advance\textwidth by \oddsidemargin
\newlength\mylength
\mylength=\textwidth
\divide\mylength by 2
\oddsidemargin=-1in
\evensidemargin=-1in
\marginparwidth=0pt
\marginparsep=0pt
\begin{document}
\input epsf
\huge

\begin{center}
2019年度卒業論文
%皆さんはここに「20XX年度卒業論文」と入れます.「」は入れませんよ.
\end{center}

\vspace*{3\baselineskip}

\begin{flushright}
\scalebox{1}{\Huge 2019年度}\\
\scalebox{1}{\Huge クラスタサイズ調整変数を導入した}\\
\scalebox{1}{\Huge クラスタリング手法の特性比較及び精度評価}
%皆さんはここに卒業論文の題目を入れます.
\end{flushright}

\vspace*{3\baselineskip}

\begin{flushright}
\begin{tabular}{l}
芝浦工業大学~~工学部\\
 通信工学科~~情報数理工学研究室\\
  指導教員: 神澤雄智
\end{tabular}
\end{flushright}

\vspace*{\baselineskip}

\begin{flushright}
\scalebox{1}{\Huge AF16009~~池辺~颯一}
%皆さんはここに学籍番号と自分の氏名を入れます.
\end{flushright}

\vspace*{\baselineskip}

\begin{flushright}
\today
%ここは最終原稿の日付を入れます.
\end{flushright}

\vspace*{-1cm}

\begin{flushleft}
\epsfxsize=\mylength
\epsfbox{simple_kanz2s2.eps}
%皆さんはここに自分の研究に関わる画像を入れます.
\end{flushleft}

\vspace*{-0.7cm}

\noindent\hfill\fbox{%
{\Huge{\color{black}K\kern-0.3ex\raisebox{0.33ex}{a\kern-0.6ex n}}\kern-0.82ex\raisebox{0.33ex}{{\color{white}\rule[-0.1ex]{0.3ex}{0.5ex}}}\kern0.25ex{\color{black}Z}\kern-1.2ex{\color{white}\rule[1.01ex]{0.4ex}{0.5ex}}\kern-2ex{\color{black}\rule[1.509ex]{2.5ex}{0.068ex}}}\mbox{ }}
%ここは変えないで下さい.
\end{document}

