%%%%%%%%%%%%%%%%%%%%%%%%%%%%%%%%%
%
%このファイルは3回のコンパイルの後,
%同梱の./listofcontents.shを用いて
%./listofcontents.sh ファイル名からドットと拡張子を除いたもの
%を1回実行し,最後にもう一度コンパイルをする必要があります.
%
%また,同梱のkanzawa.sty, kanzawa_m.sty, kanzawa2.sty, moreverb.*,
%listofcontents.sh, here.sty, jtygm.sty,
%も必要になります.
%
%
%%%%%%%%%%%%%%%%%%%%%%%%%%%%%%%%%%%
\documentclass[a4j,12pt,dvipdfmx,oneside]{jsbook}
\usepackage{kanzawa2}
\usepackage{here}
\usepackage{times}
\usepackage{amsmath,amsthm,amssymb}
\theoremstyle{definition}
\newtheorem{theorem}{定理}
\newtheorem*{theorem*}{定理}
\newtheorem{Def}[theorem]{定義}
\newtheorem*{Def*}{定義}
\newtheorem{Alg}[theorem]{アルゴリズム}
\newtheorem*{Alg*}{アルゴリズム}
\usepackage{mathptmx}
\usepackage{pifont}
\usepackage{tascmac}
\usepackage{oldgerm}
\usepackage{jtygm}
\usepackage{type1cm}
\setlength{\textwidth}{\fullwidth}
\newcommand{\vd}{\mbox{d}}
\def\rthree{I\kern-2ptI\kern-2ptI}
\newcommand{\namelistlabel}[1]{\mbox{#1}\hfil}
\newenvironment{namelist}[1]{
	\begin{list}{}
		{\let\makelabel\namelistlabel
		\settowidth{\labelwidth}{#1}
		\setlength{\leftmargin}{1.1\labelwidth}}
}{
\end{list}}

\usepackage[T1]{fontenc}
\usepackage{textcomp}
\usepackage[utf8]{inputenc}
\usepackage{lmodern}
\usepackage[all, warning]{onlyamsmath}
\kanjiencoding{JY1}
\kanjifamily{mc}
\kanjiseries{m}
\kanjishape{n}
\selectfont
\usepackage{enumitem}
\newcommand{\argmax}{\operatornamewithlimits{\mathrm{arg\,max}}}
\newcommand{\argmin}{\operatornamewithlimits{\mathrm{arg\,min}}}
\newcommand{\QED}{\hfill$\blacksquare$\par}
\usepackage{longtable}
\usepackage[dvipdfmx]{graphicx}
\usepackage{mediabb}
\makeatletter
\def\minimize{\mathop{\operator@font minimize}} 
\makeatother
\usepackage{eclbkbox}	% required for `\breakbox' (yatex added)
\begin{document}
\pagestyle{headings}
\def\thepage{\roman{page}}
\input epsf
\tableofcontents
\listoffigures
\listoftables
%%%%%
\newpage
\pagestyle{myheadings}
%
%
%
\chapter{序論}
\def\thepage{\arabic{page}}
\setcounter{page}{1}
\label{chap:first}
%
\section{背景}\label{sec:background}
%
\subsection{ファジィクラスタリング}\label{subsec:fuzzy_clustering}
%
\subsection{クラスタサイズ調整変数}\label{subsec:cluster_adjust_var}
%
\section{目的}\label{sec:purpose}
%
\section{構成}\label{sec:contents}
本文書の構成を次に示す.
第\ref{chap:before_method}章では,従来手法について説明する.
第\ref{chap:suggest_method}章では,提案手法について説明する.
第\ref{chap:artificial_data}章では,人工データ実験による各手法の特性比較を行う.
第\ref{chap:real_data}章では,実データ実験による各手法の精度比較を行う.
最後に第\ref{chap:conclusion}章では,本文書の結論を述べる.
また,付録では,プログラムソースを掲載している.
%
%
%
\chapter{従来手法}\label{chap:before_method}
%
\section{はじめに}\label{sec:before_method_intro}
本章では,卒業論文概要書や卒業論文などの作成した文書を指導教員に校正させ
る上での注意点が書かれています.
まず第~\ref{sec:checkManuscript_abst}節でその概要を示し,
次に第~\ref{sec:checkManuscript_misc}節でその他の注意点を示しています.
%
\section{おわりに}\label{sec:before_method_summary}
本章では,卒業論文概要書や卒業論文などの作成した文書を指導教員に校正させ
る上での注意点を述べました.
まず第~\ref{sec:checkManuscript_abst}節でその概要を示し,
次に第~\ref{sec:checkManuscript_misc}節でその他の注意点を示しました.
%
%
%
\chapter{提案手法}\label{chap:suggest_method}
%
\section{はじめに}\label{sec:suggest_method_intro}
本章では,卒業論文概要書の作成について述べます.
まず第\ref{sec:abst_feature}節で卒業論文概要書の性格を示します.
次に第\ref{sec:abst_flow}節で作業の流れを示します.
最後に第\ref{sec:abst_file}節でサンプルファイルの扱いを示します.
%
\section{おわりに}\label{sec:suggest_method_summary}
本章では,卒業論文概要書の作成について述べました.
まず第\ref{sec:abst_feature}節で卒業論文概要書の性格を示しました.
次に第\ref{sec:abst_flow}節で作業の流れを示しました.
最後に第\ref{sec:abst_file}節でサンプルファイルの扱いを示しました.
%
%
%
\chapter{人工データによる実験}\label{chap:artificial_data}
%
\section{はじめに}\label{sec:artificial_data_intro}
本章では,卒業論文の作成について述べます.
まず第\ref{sec:thesis_contents}節で卒業論文の構成を示します.
次に第\ref{sec:thesis_examples}節で卒業論文をより充実したものにするため
の工夫について示します.
最後に第\ref{sec:thesis_file}節でサンプルファイルの扱いを示します.
%
\section{おわりに}\label{sec:artificial_data_summary}
本章では,卒業論文の作成について述べました.
まず第\ref{sec:thesis_contents}節で卒業論文の構成を示しました.
次に第\ref{sec:thesis_examples}節で卒業論文をより充実したものにするため
の工夫について示しました.
最後に第\ref{sec:thesis_file}節でサンプルファイルの扱いを示しました.
%
%
%
\chapter{実データによる実験}\label{chap:real_data}
%
\section{はじめに}\label{sec:real_data_presen}
本章では,発表資料の作成について述べまます.
まず第\ref{sec:presen_intro}節で発表資料作成における理想的な心構えを示します.
次に第\ref{sec:presen_make}節で発表資料の理想的な作成手順を示します.
%
\section{おわりに}\label{sec:real_data_summary}
本章では,発表資料の作成について述べました.
まず第\ref{sec:presen_intro}節で発表資料作成における理想的な心構えを示し
ました.
次に第\ref{sec:presen_make}節で発表資料の理想的な作成手順を示しました.
%
%
%
%
\chapter{結論}\label{chap:conclusion}
本文書では,
第\ref{chap:before_method}章では,従来手法について説明した.
第\ref{chap:suggest_method}章では,提案手法について説明した.
第\ref{chap:artificial_data}章では,人工データ実験による各手法の特性比較を行った.
第\ref{chap:real_data}章では,実データ実験による各手法の精度比較を行った.
最後に第\ref{chap:conclusion}章では,本文書の結論を述べた.
また,付録では,プログラムソースを掲載した.
%
%
%
%
%
%
%\chapter{フォントのテスト}
%%
%%
%%
%\section{pifont}
%%
%%
%%
%\ding{"2E}\ding{"21}\Pisymbol{psy}{"A9}
%%
%%
%%
%\section{tascmac}
%%
%%
%%
%\keytop{A}\Return
%%
%%
%%
\begin{thebibliography}{5}
\bibitem{eFCMA}Ichihashi, H., Honda, K., Tani, N.: ``Gaussian Mixture PDF Approximation and Fuzzy c-means Clustering with Entropy Regularization'', Proc.~4th Asian Fuzzy System Symposium, pp.~217--221, (2000).
\bibitem{sFCMA}Miyamoto, S., Kurosawa, N.: ``Controlling Cluster Volume Sizes in Fuzzy c-means Clustering'', Proc.~SCIS\&ISIS2004, pp.~1--4, (2004).
\bibitem{qFCMA}Miyamoto, S., Ichihashi, H., and Honda, K.: Algorithms for Fuzzy Clustering, Springer (2008).
\bibitem{cFunc}宮本 定明, 馬屋原 一孝, 向殿 政男:``ファジイ $c$-平均法とエントロピー正則化法におけるファジィ分類関数,''  日本ファジィ学会誌 Vol.~10, No.~3  pp.~548--557, (1998).
\bibitem{ARI} Hubert, L., and Arabie, P.:``Comparing Partitions,'' Journal of Classifcation, Vol.~2, No.~1,
pp.~193--218, (1985).
\end{thebibliography}
%
\chapter*{感想}
\addcontentsline{toc}{chapter}{感想}\label{chap:feel}
おいしかった.
\chapter*{謝辞}
\addcontentsline{toc}{chapter}{謝辞}\label{chap:ack}
ありがとう.
\appendix
\pagestyle{headings}
\addtocontents{toc}{\protect\diamondeleaders\par}
\chapter{プログラムソース}\label{chap:program}
\section*{\texttt{runge1.c}}
\addcontentsline{toc}{section}{\texttt{runge1.c}}
\listinginput{1}{runge1.c}
\section*{\texttt{runge1.c}}
\addcontentsline{toc}{section}{\texttt{runge1.c}}
\listinginput{1}{runge1.c}
\end{document}

