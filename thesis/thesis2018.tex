%%%%%%%%%%%%%%%%%%%%%%%%%%%%%%%%%
%
%このファイルは3回のコンパイルの後,
%同梱の./listofcontents.shを用いて
%./listofcontents.sh ファイル名からドットと拡張子を除いたもの
%を1回実行し,最後にもう一度コンパイルをする必要があります.
%
%また,同梱のkanzawa.sty, kanzawa_m.sty, kanzawa2.sty, moreverb.*,
%listofcontents.sh, here.sty, jtygm.sty,
%も必要になります.
%
%
%%%%%%%%%%%%%%%%%%%%%%%%%%%%%%%%%%%
\documentclass[a4j,12pt,dvipdfmx,oneside]{jsbook}
\usepackage{kanzawa2}
\usepackage{here}
\usepackage{times}
\usepackage{amsmath,amsthm,amssymb}
\theoremstyle{definition}
\newtheorem{theorem}{定理}
\newtheorem*{theorem*}{定理}
\newtheorem{Def}[theorem]{定義}
\newtheorem*{Def*}{定義}
\newtheorem{Alg}[theorem]{アルゴリズム}
\newtheorem*{Alg*}{アルゴリズム}
\usepackage{mathptmx}
\usepackage{pifont}
\usepackage{tascmac}
\usepackage{oldgerm}
\usepackage{jtygm}
\usepackage{type1cm}
\setlength{\textwidth}{\fullwidth}
\newcommand{\vd}{\mbox{d}}
\def\rthree{I\kern-2ptI\kern-2ptI}
\newcommand{\namelistlabel}[1]{\mbox{#1}\hfil}
\newenvironment{namelist}[1]{
	\begin{list}{}
		{\let\makelabel\namelistlabel
		\settowidth{\labelwidth}{#1}
		\setlength{\leftmargin}{1.1\labelwidth}}
}{
\end{list}}

\usepackage[T1]{fontenc}
\usepackage{textcomp}
\usepackage[utf8]{inputenc}
\usepackage{lmodern}
\usepackage[all, warning]{onlyamsmath}
\kanjiencoding{JY1}
\kanjifamily{mc}
\kanjiseries{m}
\kanjishape{n}
\selectfont
\usepackage{enumitem}
\newcommand{\argmax}{\operatornamewithlimits{\mathrm{arg\,max}}}
\newcommand{\argmin}{\operatornamewithlimits{\mathrm{arg\,min}}}
\newcommand{\QED}{\hfill$\blacksquare$\par}
\usepackage{longtable}
\usepackage[dvipdfmx]{graphicx}
\usepackage{mediabb}
\makeatletter
\def\minimize{\mathop{\operator@font minimize}} 
\makeatother
\usepackage{eclbkbox}	% required for `\breakbox' (yatex added)
\begin{document}
\pagestyle{headings}
\def\thepage{\roman{page}}
\input epsf
\tableofcontents
\listoffigures
\listoftables
%%%%%
\newpage
\pagestyle{myheadings}
%
%
%
\chapter{序論}
\def\thepage{\arabic{page}}
\setcounter{page}{1}
\label{chap:first}
%
\section{背景}\label{sec:background}
%


\subsection{ファジィクラスタリング}\label{subsec:fuzzy_clustering}
%
\subsection{クラスタサイズ調整変数}\label{subsec:cluster_adjust_var}
%
\section{目的}\label{sec:purpose}
%
\section{構成}\label{sec:contents}
本文書の構成を次に示す.
第\ref{chap:suggest_method}章では,提案手法について説明する.
第\ref{chap:artificial_data}章では,人工データ実験による各手法の特性比較を行う.
第\ref{chap:real_data}章では,実データ実験による各手法の精度比較を行う.
最後に第\ref{chap:conclusion}章では,本文書の結論を述べる.
また,付録では,プログラムソースを掲載している.
%
%
%
\chapter{提案手法}\label{chap:suggest_method}
%
\section{はじめに}\label{sec:suggest_method_intro}
本章では,本研究で提案するファジィクラスタリング手法について説明する.
まず第~\ref{sec:suggest_method_define}節で定義を示し,
次に第~\ref{sec:efcma}節から~\ref{sec:qfcma}節で各手法の最適化問題と,その導出について述べる.
最後に第~\ref{sec:suggest_algorythm}節でアルゴリズムについて述べる.
%
\section{定義}\label{sec:suggest_method_define}
%
次節で述べるファジィクラスリングの最適化問題における各変数の定義について、以下の表に示す。
 \begin{table}[htbp]
  \caption{ファジイクラスタリングの最適化問題における定義}
 \begin{center}
  \begin{tabular}{c|c||c|c} \hline
   {$N$}&データ数&{$x_k$}&データ \\ \hline
   {$C$}&クラスタ数&{$v_i$}&クラスタ中心\\ \hline
   {$\lambda$,~$m$}&ファジィ化パラメータ&{$u_{i,k}$}&帰属度 \\ \hline
   {$\alpha_i$}&クラスタサイズ調整変数\\ \hline
  \end{tabular}
 \end{center}
  \label{tab:fuzzy_c_define}
 \end{table}

\section{eFCMA}\label{sec:efcma}
%
Entropy-regularized Fuzzy $c$-Means vAriable controlling clusters size~(eFCMA)~\cite{eFCMA}
の最適化問題を以下に示す。

\begin{align}
 \underset{u,v,\alpha}{\text{minimize}}
    \sum_{i=1}^C\sum_{k=1}^Nu_{i,k}||x_k-v_i||_2^2+\lambda^{-1}\sum_{i=1}^C\sum_{k=1}^Nu_{i,k}\log\Bigl(\frac{u_{i,k}}{\alpha_{i}}\Bigl)\\
    {\text{subject to }}\sum_{i=1}^Cu_{i,k}=1,~\sum_{i=1}^C\alpha_{i}=1{\text{ and }}\lambda>0,~\quad\alpha_{i}>0
\end{align}

次に、
\begin{align}
    v_{i}=\frac{\sum_{k=1}^Nu_{i,k}x_{k}}{\quad\sum_{k=1}^Nu_{i,k}}
\end{align}

\begin{align}
    u_{i,k}=\frac{\pi_{i}\exp(-\lambda||x_k-v_i||_2^2)}{\sum_{j=1}^C\alpha_{j}\exp(-\lambda||x_k-v_j||_2^2)}
\end{align}

\begin{align}
    \alpha_{i}=\frac{\sum_{k=1}^Nu_{i,k}}{\quad N}
\end{align}

\section{sFCMA}\label{sec:sfcma}
%
Standard Fuzzy $c$-Means with vAriable controlling cluster size~(sFCMA)~\cite{sFCMA}
の最適化問題を以下に示す。

\begin{align}
 \underset{u,v,\alpha}{\text{minimize}}
 \sum_{i=1}^C\sum_{k=1}^N(\alpha_{i})^{1-m}(u_{i,k})^m||x_k-v_i||_2^2\\
 {\text{subject to }}\sum_{i=1}^Cu_{i,k}=1,~\sum_{i=1}^C\alpha_{i}=1{\text{ and }}m>1,~\alpha_{i}>0
\end{align}
 
\begin{align}
   v_{i}=\frac{\sum_{k=1}^N(u_{i,k})^mx_{k}}{\quad\sum_{k=1}^N(u_{i,k})^{m}}
\end{align}

\begin{align}
   u_{i,k}=\frac{1}{\sum_{j=1}^c\frac{\alpha_{j}}{\alpha_{i}}\left(\frac{d_{j,k}}{d_{i,k}}\right)^\frac{1}{1-m}}
\end{align}

\begin{align}
    \alpha_{i}=\frac{1}{\sum_{j=1}^C\left(\sum_{k=1}^N\frac{(u_{j,k})^md_{j,k}}{(u_{i,k})^md_{i,k}}\right)^{\frac{1}{m}}}
\end{align}
  
\section{qFCMA}\label{sec:qfcma}
%
$q$-divergence based Fuzzy $c$-Means with vAriable controlling cluster size~(qFCMA)~\cite{qFCMA}
の最適化問題を以下に示す。

\begin{align}
 \underset{u,v,\alpha}{\text{minimize}}
 \sum_{i=1}^C\sum_{k=1}^N(\alpha_{i})^{1-m}(u_{i,k})^m||x_k-v_i||_2^2
 +\frac{\lambda^{-1}}{m-1}\sum_{i=1}^C\sum_{k=1}^N(\alpha_{i})^{1-m}(u_{i,k})^m\\
 {\text{subject to }}\sum_{i=1}^Cu_{i,k}=1,~\sum_{i=1}^C\alpha_{i}=1{\text{ and }}\lambda>0,~m>1,~\alpha_{i}>0
\end{align}

\begin{align}
  v_{i}=\frac{\sum_{k=1}^N(u_{i,k})^mx_{k}}{\quad\sum_{k=1}^N(u_{i,k})^{m}}
\end{align}
    
\begin{align}
  u_{i,k}=\frac{\alpha_{i}(1+\lambda(1-m)||x_i-v_k||_2^2)^\frac{1}{1-m}}{\quad\sum_{j=1}^C\alpha_{j}(1+\lambda(1-m)||x_j-v_k||_2^2)^\frac{1}{1-m}}
\end{align}

\begin{align}
 \alpha_{i}=\frac{1}{\sum_{j=1}^C\left(\sum_{k=1}^N\frac{(u_{j,k})^m(1-\lambda(1-m)d_{j,k})}{(u_i,k)^m(1-\lambda(1-m)d_{i,k})}\right)^{\frac{1}{m}}}
\end{align}
   
\section{アルゴリズム}\label{sec:suggest_algorythm}
%
\section{おわりに}\label{sec:suggest_method_summary}
本章では,本研究で提案するファジィクラスタリング手法について説明した.
まず第~\ref{sec:suggest_method_define}節で定義を示し,
次に第~\ref{sec:efcma}節から~\ref{sec:qfcma}節で各手法の最適化問題と,その導出について述べた.
最後に第~\ref{sec:suggest_algorythm}節でアルゴリズムについて述べた.
%
%
%
\chapter{人工データによる実験}\label{chap:artificial_data}
%
\section{はじめに}\label{sec:artificial_data_intro}
本章では,人工データを用いた実験について述べる.
まず第\ref{sec:about_artificial_data}節で本実験で用いる人工データについて説明する.
次に第\ref{sec:classification_function}節で実験により得られた分類関数を用いて各手法の特性比較を行う.
%
\section{人工データについて}\label{sec:about_artificial_data}
%
\section{分類関数による特性比較}\label{sec:classification_function}
%
\section{おわりに}\label{sec:artificial_data_summary}
本章では,人工データを用いた実験について述べた.
まず第\ref{sec:about_artificial_data}節で本実験で用いる人工データについて説明した.
次に第\ref{sec:classification_function}節で実験により得られた分類関数を用いて各手法の特性比較を行った.
%
%
%
\chapter{実データによる実験}\label{chap:real_data}
%
\section{はじめに}\label{sec:real_data_intro}
本章では,実データを用いた実験について述べる.
まず第\ref{sec:about_real_data}節で本実験で用いる実データについて説明する.
次に第\ref{sec:ari_compare}節で実験により得られた評価指標を用いて各手法の精度比較を行う.
%
\section{実データについて}\label{sec:about_real_data}
%
\section{ARIによる精度比較}\label{sec:ari_compare}
%
\section{おわりに}\label{sec:real_data_summary}
本章では,実データを用いた実験について述べた.
まず第\ref{sec:about_real_data}節で本実験で用いる人工データについて説明した.
次に第\ref{sec:ari_compare}節で実験により得られた評価指標を用いて各手法の精度比較を行った.
%
%
%
%
\chapter{結論}\label{chap:conclusion}
本文書では,
第\ref{chap:suggest_method}章では,提案手法について説明した.
第\ref{chap:artificial_data}章では,人工データ実験により各手法の特性比較を行った.
第\ref{chap:real_data}章では,実データ実験により各手法の精度比較を行った.
最後に第\ref{chap:conclusion}章では,本文書の結論を述べた.
また,付録では,プログラムソースを掲載した.
%
%
%
%
%
%
%\chapter{フォントのテスト}
%%
%%
%%
%\section{pifont}
%%
%%
%%
%\ding{"2E}\ding{"21}\Pisymbol{psy}{"A9}
%%
%%
%%
%\section{tascmac}
%%
%%
%%
%\keytop{A}\Return
%%
%%
%%
\begin{thebibliography}{5}
\bibitem{eFCMA}Ichihashi, H., Honda, K., Tani, N.: ``Gaussian Mixture PDF Approximation and Fuzzy c-means Clustering with Entropy Regularization'', Proc.~4th Asian Fuzzy System Symposium, pp.~217--221, (2000).
\bibitem{sFCMA}Miyamoto, S., Kurosawa, N.: ``Controlling Cluster Volume Sizes in Fuzzy c-means Clustering'', Proc.~SCIS\&ISIS2004, pp.~1--4, (2004).
\bibitem{qFCMA}Miyamoto, S., Ichihashi, H., and Honda, K.: Algorithms for Fuzzy Clustering, Springer (2008).
\bibitem{cFunc}宮本 定明, 馬屋原 一孝, 向殿 政男:``ファジイ $c$-平均法とエントロピー正則化法におけるファジィ分類関数,''  日本ファジィ学会誌 Vol.~10, No.~3  pp.~548--557, (1998).
\bibitem{ARI} Hubert, L., and Arabie, P.:``Comparing Partitions,'' Journal of Classifcation, Vol.~2, No.~1,
pp.~193--218, (1985).
\end{thebibliography}
%
\chapter*{感想}
\addcontentsline{toc}{chapter}{感想}\label{chap:feel}
おいしかった.
\chapter*{謝辞}
\addcontentsline{toc}{chapter}{謝辞}\label{chap:ack}
ありがとう.
\appendix
\pagestyle{headings}
\addtocontents{toc}{\protect\diamondeleaders\par}
\chapter{プログラムソース}\label{chap:program}
\section*{\texttt{vector.h}}
\addcontentsline{toc}{section}{\texttt{vector.h}}
\listinginput{1}{src/vector.h}
\section*{\texttt{vector.cxx}}
\addcontentsline{toc}{section}{\texttt{vector.cxx}}
\listinginput{1}{src/vector.cxx}
\section*{\texttt{matrix.h}}
\addcontentsline{toc}{section}{\texttt{matrix.h}}
\listinginput{1}{src/matrix.h}
\section*{\texttt{matrix.cxx}}
\addcontentsline{toc}{section}{\texttt{matrix.cxx}}
\listinginput{1}{src/matrix.cxx}
\end{document}

